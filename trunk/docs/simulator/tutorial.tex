\documentclass[12pt]{article}

\usepackage{fullpage}

\author{Marc G. Bellemare, Adam White}
\title{The RLAI Robotic Simulator\\ A Tutorial}

\begin{document}
\maketitle

\section{Requirements}

\begin{itemize}
\item{Java 1.5 (1.6?)}
\end{itemize}

I suggest running the simulator locally, at least initially, to be able
to visualize the environment.

\section{Installing the Simulator}

Installing the simulator is a matter of a few steps. In order, you should:

\begin{enumerate}
\item{Download and build the DisCo framework}
\item{Download and extract the RLAI Robotic Simulator package}
\item{Copy relevant files into the Simulator package}
\end{enumerate}

Each step will be detailed in the following sections.


\subsection{Building DisCo}

According to its website, DisCo is a ``framework for developing and running 
applications that must process data which is being produced in real time.'' 
For now we will not
concern with the details of what the framework does; you can learn more at
\begin{verbatim}http://www.cs.ualberta.ca/~roberts/intro.html\end{verbatim}.

Thankfully, you can download a pre-packaged version of DisCo, extended with
Critterbot-specific files, at 

\begin{verbatim}
http://www.cs.ualberta.ca/~mg17/critterbot/files/DiscoExtension.tgz
\end{verbatim}

Next, you need to copy a file specifying some information about your operating
system. The file in question is:

\begin{itemize}
\item{\textbf{Mac OS}: make/LIBSMAC.d}
\item{\textbf{Unix}: make/LIBSMAC.d (untested)}
\item{\textbf{Windows}: Out of luck.}
\end{itemize}

To copy the file, execute (from your disco directory): 

\begin{verbatim}$> cp make/LIBSMAC.d make/LIBS.d \end{verbatim}

We are now ready to build:
\begin{verbatim}
$> cd ../.. 
$> make
\end{verbatim}

\subsection{Downloading and installing the RLAI Robotic Simulator package}

The Simulator package can be downloaded from

\begin{verbatim}
http://www.cs.ualberta.ca/~mg17/critterbot/files/RLAIRobotSimulator.tgz
\end{verbatim}

You should be able to directly extract the directory, e.g.

\begin{verbatim}$> tar -xvzf RLAIRobotSimulator.tgz \end{verbatim}

under UNIX systems.

\subsection{Copying relevant files to the RLAI Robotic Simulator package}

There is a single file of interest in the DisCo directory, located in the
\verb+bin+ directory and named \verb+Control.bin+. If you extracted the 
DisCo code to its default directory, \verb+disco+, then the file should 
be located at \verb+disco/bin/Control.bin+.

All that is needed is to copy the file over into the Simulator directory.
Assuming the installation directory for the Simulator is \verb+simulator+,
then you can copy the file as follows:

\begin{verbatim}$> cp disco/bin/Control.bin simulator/bin/Control.bin \end{verbatim}

(For UNIX users, ln -s is recommended)

\section{Running the Simulator}

Provided you installed all the required files, you should now be able to
run the Simulator out of the box:

\begin{verbatim}$> cd simulator \end{verbatim}
\begin{verbatim}$> ./run.sh \end{verbatim}

The \verb+run.sh+ script will compile three Java files, namely

\begin{itemize}
\item{\verb+src/Simulator.java+}
\item{\verb+src/Agent.java+}
\item{\verb+src/RewardGenerator.java+}
\end{itemize}

These files respectively create the simulator, a simple agent and a reward
generator. By default they open up ports 2324, 2325 and 2326. You can modify
these ports (see Section \ref{sec:changingDefaults} for more information).

After compiling the Java sources, the script will run the corresponding
binaries and the DisCo binary \verb+Control.bin+. The latter is in charge
of the communication between the three components. The script will then wait
for any of the four processes to die, at which point it will attempt to 
terminate all of them.

Once you start \verb+run.sh+, you should see a GUI appear (provided you are
running locally). By default the Agent will send commands at period intervals;
you should see the robot performing some behavior which depends on the mood
of the programmer at the time the simulator was packaged. You can also
manually control the robot with the arrow keys within the GUI; however, this
will result in the Agent and yourself competing for control, which is probably
not the desired behavior. If you wish to play around the simulator without
the Agent, you can run the \verb+Simulator+ process on its own (peek inside
\verb+run.sh+ for more details).

% @todo add GUI image
\end{document}
