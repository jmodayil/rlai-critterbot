\documentclass[12pt]{article}

\usepackage{fullpage}
\usepackage{epsfig}
\usepackage{graphicx}

\author{Marc G. Bellemare, Adam White}
\title{The RLAI Robotic Simulator\\ A Tutorial}

\begin{document}
\maketitle

\section{Introduction}

This tutorial documents the RLAI Robotic Simulator: how to install it, how
to run it and how to use it in your work. After installing the simulator
you will have a choice of operating it in Normal mode (involving an agent
and a reward function) or in Standalone mode, where you can personally 
control the robot in the environment. After performing the installation
(Section \ref{sec:installation}),  I suggest running the simulator in 
Standalone mode (see Section \ref{subsec:standalone}) once initially to get
an idea of the environment in which the robot evolves. 

\section{Requirements}

\begin{itemize}
\item{Java 1.5}
\item{A POSIX Operating System}
\end{itemize}

\section{Installing the Simulator}\label{sec:installation}

Installing the simulator is a matter of a few steps. In order, you should:

\begin{enumerate}
\item{Download the Simulator package}
\item{Build the DisCo framework}
\item{Copy relevant files into the Simulator}
\end{enumerate}

Each step will be detailed in the following sections with instructions for
UNIX systems.

\subsection{Downloading and installing the RLAI Robotic Simulator}

The Simulator package can be downloaded from

\begin{verbatim}
http://www.cs.ualberta.ca/~mg17/critterbot/files/RLAISimulator.tgz
\end{verbatim}

For the remaining of this tutorial, I will assume that you have downloaded
this file to a directory called \verb+robots+. Assuming that this is done, 
you should extract files from the tarball:

\begin{verbatim}robots> tar -xvzf RLAISimulator.tgz \end{verbatim}

The simulator itself, being in Java, does not need to be compiled. Before
we actually use it, however, we need to set up the framework, called DisCo,
that binds the simulator, the agent and the reward generator together.

\subsection{Building DisCo}

According to its website, DisCo is a ``framework for developing and running 
applications that must process data which is being produced in real time.'' 
For now we will not
concern with the details of what the framework does; you can learn more at
\begin{verbatim}http://www.cs.ualberta.ca/~roberts/intro.html\end{verbatim}.

Thankfully, \verb+RLAISimulator.tgz+ comes with a pre-packaged version of 
DisCo, extended with Critterbot-specific files. The files of interest are in
a subdirectory \verb+disco+. Before compiling the framework, you need to copy 
a file specifying some information about your operating system (DisCo is meant
to be cross-platform). Depending on your operation system, the file in 
question is:

\begin{itemize}
\item{\textbf{Linux}: LIBSLINUX.d}
% UNIX?
\item{\textbf{Mac OS}: LIBSMAC.d}
\item{\textbf{Windows}: Out of luck.}
\end{itemize}

Assuming a Linux machine, execute the following:

\begin{verbatim}
robots> cd disco
robots/disco> cp make/LIBSLINUX.d make/LIBS.d
\end{verbatim}

On a Mac OS-based machine, you would replace LIBSLINUX.d by LIBSMACS.d. We 
are now ready to build:

\begin{verbatim}
robots/disco> make

<crunch crunch crunch>

robots/disco> cd ..
robots>
\end{verbatim}


\subsection{Copying relevant files to the RLAI Robotic Simulator}

There is a single file of interest in the DisCo directory, located in the
\verb+bin+ directory and named \verb+Control.bin+. If you extracted the 
DisCo code to its default directory, \verb+disco+, then the file should 
be located at \verb+disco/bin/Control.bin+.

All that is needed is to copy the file over into the Simulator directory.
Assuming the installation directory for the Simulator is \verb+simulator+,
then you can copy the file as follows:

\begin{verbatim}
robots> cp disco/bin/Control.bin simulator/bin/Control.bin
\end{verbatim}

This completes the installation! 

\section{Running the Simulator}

Provided that you installed all the required files, you should now be able to
run the Simulator out of the box. There are two possible modes of operation,
namely standalone mode and normal mode. In standalone mode only the simulator
itself will be running, without an agent or a source of rewards. In this mode
you can control the robot manually using the arrow keys. On the other hand,
in normal mode the Agent program (\verb+src/Agent.java+) controls the robot.

Playing with the simulator in standalone mode can give you an idea of how the
robot moves around and how objects react before designing an agent.

\subsection{Standalone Mode}\label{subsec:standalone}

The following will start the simulator in standalone mode:

\begin{verbatim}
robots> cd simulator 
robots/simulator> ./runStandalone.sh
\end{verbatim}

The \verb+runStandalone.sh+ script will compile the simulator loader,
\verb+src/Simulator.java+, and execute it. The specifics of this program will 
be detailed later in this tutorial.

Once you execute \verb+runStandalone.sh+, you should see a GUI appear, showing
the state of the simulator. The arrow keys will drive the robot around. Some 
of the
sensory information is currently drawn (such as bump sensors), but we do not
aim to provide a complete visualization of the data.

\subsection{Normal Mode}

The following will start the simulator in normal mode:

\begin{verbatim}
robots> cd simulator 
robots/simulator> ./run.sh
\end{verbatim}

The \verb+run.sh+ script will compile three Java files, namely

\begin{itemize}
\item{\verb+src/Simulator.java+}
\item{\verb+src/Agent.java+}
\item{\verb+src/RewardGenerator.java+}
\end{itemize}

These files respectively create the simulator, a simple agent and a reward
generator. By default they open up ports 2324, 2325 and 2326. You can modify
these ports (see Section \ref{sec:changingDefaults} for more information).

After compiling the Java sources, the script will run the corresponding
binaries and the DisCo binary \verb+Control.bin+. The latter is in charge
of the communication between the three components. The script will then wait
for any of the four processes to die, at which point it will attempt to 
terminate all of them.

Once you start \verb+run.sh+, you should see a GUI appear (provided you are
running locally). By default the Agent will send commands at regular intervals;
you should see the robot performing some behavior which depends on the mood
of the programmer at the time the simulator was packaged. The arrow keys are
disabled in Normal Mode to prevent competition for control between the agent
and the user. 

\begin{figure}
\centerline{
\psfig{file=images/simulator_gui.pdf,width=5in}
}
\caption{The RLAI Robotic Simulator.}
\end{figure}

\section{Default Parameters and Command-Line Arguments\label{sec:changingDefaults}}

\subsection{Simulator}\label{subsec:simulator_parameters}

The following command-line arguments can be used with \verb+Simulator.java+:

\begin{center}
\begin{tabular}{|c|l|}
\hline
\verb+-p [port]+ & specifies which port the Simulator should listen on. Default: 2324. \\
\hline
\verb+-ng+ & disables the GUI (and keyboard control). \\
\hline
\verb+-nk+ & disables controlling the robot via the keyboard. \\
\hline
\verb+-s [scale]+ & specifies the time scale (see below for details). Default: 1.0. \\
\hline
\verb+-h+ & provides help on command-line options \\ 
\hline
\end{tabular}
\end{center}

The time scale $s$ is specified in relative speed of the simulator with respect 
to wall clock. If $s=1.0$, 1 second of simulator time is simulated in 1 second
of wall time. If $s=0.5$, 0.5 second of simulator time is simulated in 1
second of wall time. Values larger than $1.0$ may tax the processor too much,
in which case the simulator will report `Running behind' and slow down.

\subsection{Agent}

In order to modify the port on which the Agent listens, the file 
\verb+Agent.java+ will have to be modified and the variable \verb+discoAgentPort+ changed to a different value.

\subsection{Reward Generator}

% @todo

\subsection{DisCo}

The configuration for DisCo may be found in 
\verb+robots/simulator/bin/simulator.xml+. It is beyond the scope of this
tutorial to detail how DisCo should be configured. Of chief interest are the
ports that DisCo connects to. By default these will be 2324, 2325 and 2326.
If modifying the ports for either the Simulator, Agent or Reward Generator,
the DisCo configuration file will also have to be modified accordingly.

\section{Logging and Common Error Messages}

The output of all processes (there are four of them) is redirected by default
to files in the directory \verb+simulator/logs+. In this section we review
what kind of messages you should expect to see if you take a peek at one of
the output logs.

\subsection{Simulator}

The log file for the simulator is \verb+simulator/logs/Simulator+. The 
simulator, like all pieces of software, is not perfect. The most common
error message you are likely to see is

\begin{verbatim}
Simulator is running behind...
\end{verbatim}

This is the simulator's way of telling you that it is unable to simulate
at the requested speed. The easiest way to get rid of it is to reduce the
time scale at which the simulator is asked to operate (below 1.0 if necessary).
In any event, no harm is caused; the simulator will trudge along as fast as it
can.

\subsection{DisCo}

The log file for DisCo is \verb+simulator/logs/Disco+. DisCo actually generates 
a lot of output, including information about the data
that is being passed around. A discussion on the Drop data structure and how 
it is handled by DisCo is beyond the scope of this tutorial. However, most of 
the data in the DisCo log file should be easily interpreted.

\section{Final Notes}

The provided package is meant to be easy to install and use. Most likely,
however, you will want to run the Agent (and maybe the RewardGenerator) in
a separate console in order to more easily debug it.

If you have any concerns or questions about the package or the simulator,
you should contact Marc G. Bellemare (\texttt{mgbellemare \_at\_ ualberta.ca+}) or Adam White (\texttt{awhite \_at\_ cs.ualberta.ca}).

\end{document}
